% Options for packages loaded elsewhere
\PassOptionsToPackage{unicode}{hyperref}
\PassOptionsToPackage{hyphens}{url}
%
\documentclass[
]{article}
\usepackage{lmodern}
\usepackage{amssymb,amsmath}
\usepackage{ifxetex,ifluatex}
\ifnum 0\ifxetex 1\fi\ifluatex 1\fi=0 % if pdftex
  \usepackage[T1]{fontenc}
  \usepackage[utf8]{inputenc}
  \usepackage{textcomp} % provide euro and other symbols
\else % if luatex or xetex
  \usepackage{unicode-math}
  \defaultfontfeatures{Scale=MatchLowercase}
  \defaultfontfeatures[\rmfamily]{Ligatures=TeX,Scale=1}
\fi
% Use upquote if available, for straight quotes in verbatim environments
\IfFileExists{upquote.sty}{\usepackage{upquote}}{}
\IfFileExists{microtype.sty}{% use microtype if available
  \usepackage[]{microtype}
  \UseMicrotypeSet[protrusion]{basicmath} % disable protrusion for tt fonts
}{}
\makeatletter
\@ifundefined{KOMAClassName}{% if non-KOMA class
  \IfFileExists{parskip.sty}{%
    \usepackage{parskip}
  }{% else
    \setlength{\parindent}{0pt}
    \setlength{\parskip}{6pt plus 2pt minus 1pt}}
}{% if KOMA class
  \KOMAoptions{parskip=half}}
\makeatother
\usepackage{xcolor}
\IfFileExists{xurl.sty}{\usepackage{xurl}}{} % add URL line breaks if available
\IfFileExists{bookmark.sty}{\usepackage{bookmark}}{\usepackage{hyperref}}
\hypersetup{
  pdftitle={Productivity Model Equations},
  pdfauthor={J.R. Blaszczak, C.B. Yackulic, R.K. Shriver, R.O. Hall, Jr.},
  hidelinks,
  pdfcreator={LaTeX via pandoc}}
\urlstyle{same} % disable monospaced font for URLs
\usepackage[margin=1in]{geometry}
\usepackage{graphicx,grffile}
\makeatletter
\def\maxwidth{\ifdim\Gin@nat@width>\linewidth\linewidth\else\Gin@nat@width\fi}
\def\maxheight{\ifdim\Gin@nat@height>\textheight\textheight\else\Gin@nat@height\fi}
\makeatother
% Scale images if necessary, so that they will not overflow the page
% margins by default, and it is still possible to overwrite the defaults
% using explicit options in \includegraphics[width, height, ...]{}
\setkeys{Gin}{width=\maxwidth,height=\maxheight,keepaspectratio}
% Set default figure placement to htbp
\makeatletter
\def\fps@figure{htbp}
\makeatother
\setlength{\emergencystretch}{3em} % prevent overfull lines
\providecommand{\tightlist}{%
  \setlength{\itemsep}{0pt}\setlength{\parskip}{0pt}}
\setcounter{secnumdepth}{-\maxdimen} % remove section numbering

\title{Productivity Model Equations}
\author{J.R. Blaszczak, C.B. Yackulic, R.K. Shriver, R.O. Hall, Jr.}
\date{}

\begin{document}
\maketitle

This document is an overview of the progress thus far on modeling
productivity dynamics in rivers. All code and data is available at this
Github page: \url{https://github.com/jrblaszczak/RiverBiomass}

\hypertarget{methods-overview}{%
\subparagraph{Methods Overview}\label{methods-overview}}

We estimated productivity dynamics in rivers through time with five
stochastic state-space models that account for process and observation
error. We fit each model to annual time series of oxygen-derived daily
gross primary productivity (GPP) rates. The first model is
phenomenological, meaning it does not infer causality, but instead
approximates lags through an autoregressive parameter, a positive
correlative relationship between GPP and light indicative of autotrophic
biomass growth, and a negative relationship between GPP and discharge
indicative of autotrophic biomass loss. The next four models are
semi-mechanistic, meaning they are meant to represent the mechanisms by
which productivity rates may increase or decrease through time through
the growth and detachment of autotrophic biomass. Therefore, these four
models include a latent variable ``biomass'' meant to represent the
amount of autotrophic biomass contributing to the diel variation in the
dissolved oxygen signal from which daily GPP is estimated. The
semi-mechanistic models differ in the structure of the equation meant to
represent biomass growth dynamics, but share a similar disturbance
component, both of which are described below.

We selected a subset of river metabolism time series from 356 rivers
with daily estimates of metabolism generated using the streamMetabolizer
package in R by the US Geological Survey Powell Center (Appling et
al.~2018b). Daily metabolism estimates were generated using a
hierarchical state-space inverse modeling approach with partial-pooling
of piece-wise \(K_{600}\) relationships with mean daily discharge
(\(Q\)) to reduce issues of equifinality and uncertainty (Appling et
al.~2018a). Daily metabolism and gas exchange estimates were generated
from sub-daily time series of dissolved oxygen (units: \(mg\)
\(L^{-1}\)), light (photosynthetic photon flux density (PPFD); units:
\(\mu mol\ m^{-2}\ s^{-1}\)), water temperature (\(^\circ\) C), and
reach averaged depth (m). A Bayesian Markov chain Monte Carlo (MCMC)
fitting procedure was used to determine the mean and standard deviation
of the posterior probability distributions of daily gross primary
productivity (GPP; \(g\ O_{2}\ m^{-2}\ d^{-1}\)), ecosystem respiration
(ER; \(g\ O_{2}\ m^{-2}\ d^{-1}\)), and gas exchange (\(K_{600}\);
\(d^{-1}\)). In this study, we treat the mean and standard deviation of
the daily GPP estimates from each river as ``data'' in the state-space
models described below.

\textbf{There are three sections to this document:}

\begin{enumerate}
\def\labelenumi{\arabic{enumi}.}
\item
  Written descriptions of each model.
\item
  Figures comparing model simulations using the posterior predictives
  from each model using incoming light, and a second version of the same
  model which has another parameter to attempt estmation of benthic
  light after accounting for attenuation from turbidity. All models are
  fit to GPP time series from 2010 in the Clackamas River, Oregon. Links
  to each stan file and each simulation function are included. This
  section also includes a comparison of the daily RMSE among models (DIC
  comparison forthcoming).
\item
  Simulations from posteriors and parameter recovery for the Ricker
  Model without benthic light adjustments across nine rivers. The
  selection of rivers was a relatively random subset to this point
  (sites with \textgreater350 with clear visual decline in GPP in
  response to a storm), but I think a more interesting comparison will
  be to choose \textasciitilde10 river years with different flow regimes
  to demonstrate where and when the models work best.
\end{enumerate}

\hypertarget{references}{%
\subparagraph{References}\label{references}}

Appling, A.P., Hall, R.O., Yackulic, C.B. \& Arroita, M. (2018a).
Overcoming equifinality: Leveraging long time series for stream
metabolism estimation. J. Geophys. Res. Biogeosciences, 123, 624-645.

Appling, A.P., Read, J.S., Winslow, L.A., Arroita, M., Bernhardt, E.S.,
Griffiths, N.A., et al.~(2018b). The metabolic regimes of 356 rivers in
the United States. Sci. Data, 5, 180292.

\hypertarget{model-descriptions}{%
\subsection{1. Model descriptions}\label{model-descriptions}}

\hypertarget{productivity-model-1-linear-autoregressive-model}{%
\subsubsection{Productivity Model 1: Linear autoregressive
model}\label{productivity-model-1-linear-autoregressive-model}}

We first predict the temporal dynamics of daily estimates of GPP (\(g\))
using a linear autoregressive model without latent biomass dynamics. We
fit the model to the \(g\) times series from Appling et al.~(2018a) to
estimate the parameters described below:

\begin{equation} 
G_{t} \sim N(\phi G_{(t-\Delta t)}+\alpha L_{t}+\beta Q_{t}, \sigma_{proc})
\end{equation}

\begin{equation} 
g_{t} \sim N(e^{G_{t}},\sigma_{obs})
\end{equation}

where input data includes the daily mean of the posterior probability
distribution of previously modeled daily GPP time series
(\(GPP_{mod,t}\); \(g\ O_{2}\ m^{-2}\ d^{-1}\)), \(L\) which is daily
light at time \(t\) relativized to the annual maximum daily light
(unitless; 0\textless{}\(L\)\textless1), and \(Q_t\) which is daily
discharge at time \(t\) relativized to the annual maximum daily
discharge (unitless; 0\textless{}\(Q\)\textless1). Light and discharge
were relativized to the annual maximum as opposed to zero to avoid
switching between positive and negative values. Estimated parameters
include \(G_{t}\) which is the predicted daily GPP (g \(O_2\) \(m^{-2}\)
\(d^{-1}\)) at time \(t\) (in days) on a log scale, \(\phi\) which is
the estimated autoregressive parameter (unitless;
0\textless{}\(\phi\)\textless1), \(\alpha\) which is a light use
efficiency parameter of log GPP growth per unit relativized light,
\(\beta\) which is a loss parameter of log GPP loss per unit relativized
discharge, \(\sigma_{obs}\) which is the observation error set to the
standard deviation of the posterior probability distribution of the
previously modeled daily GPP estimates (\(g\)), \(\sigma_{proc}\) which
is the process error.

For the benthic light version of this model, \(L_t\) is replaced by
\(benL_t\) in Lambert's law:

\begin{equation}
benL_{t} = L_{t}e^{(-1 \times a \times T_{t} D_{t})}
\end{equation}

where \(a\) is an esimated parameter to estimate the downwelling light
attenuation coefficient (\(K_d = a \times T_{t}\)), \(T_t\) is daily
turbidity (NTU), and \(D_t\) is daily depth (m) from Appling et
al.~(2018b).

Stan code (surface light):
\url{https://github.com/jrblaszczak/RiverBiomass/blob/main/code/Stan_ProductivityModel1_Autoregressive.stan}

Stan code (benthic light):
\url{https://github.com/jrblaszczak/RiverBiomass/blob/main/code/Stan_ProductivityModel1_Autoregressive_BenthicL.stan}

Simulation code (surface and benthic light):
\url{https://github.com/jrblaszczak/RiverBiomass/blob/main/code/Simulated_ProductivityModel1_Autoregressive.R}

\hypertarget{productivity-model-2-latent-biomass-logistic-growth}{%
\subsubsection{Productivity Model 2: Latent biomass logistic
growth}\label{productivity-model-2-latent-biomass-logistic-growth}}

We predict the temporal dynamics of mean daily estimates of GPP (\(g\))
by incorporating biomass as a latent variable and modeling its dynamics
using a logistic growth model. We fit the model to the \(g\) times
series from Appling et al.~(2018b) to estimate the model parameters. We
incorporated the effects of disturbance by modeling the persistence
(\(P\)) of biomass using a complementary log-log link function of the
form:

\begin{equation}
    P_{t} = e^{-e^{s*(Q_{t} - c)}}
\end{equation}

where input data includes \(Q_t\) which is daily discharge at time \(t\)
relativized to the annual maximum. Estimated parameters include \(s\)
which is a parameter that characterizes the steepness of the persistence
transition and \(c\) is an estimated parameter which approximates the
critical discharge at which autotrophic biomass is disturbed. The
log-log link function constrains values between 0 and 1, where 0 is no
persistence and 1 is complete persistence.

Together, the full model took the form:

\begin{equation}
    B_{t} \sim N(P_{t}(B_{(t-\triangle t)} e^{r_{max}B_{(t-\triangle t)}(1-\frac{B_{(t-\triangle t)}}{K})}), \sigma_{proc})
\end{equation}

\begin{equation}
    g_{t} \sim N(L_{t} e^{B_{t}}, \sigma_{obs})
\end{equation}

where estimated parameters include \(B_{t}\) which is a latent variable
representative of photosynthetically-active biomass on a log scale and
daily time step, \(P_{t}\) which is the daily persistence of biomass
dependent on hydrologic disturbance and removal detailed above, \(K\)
which is the estimated carrying capacity of a river, \(r_{max}\) is the
maximum per capita growth rate when \(\frac {B_t}{K}\) approaches zero
following a disturbance, \(\alpha\) which is a light growth efficiency
parameter (units: g \(O_2\) \(m^{-2}\) \(d^{-1}\)), and
\(\sigma_{proc}\) which is the process error. Input data and the benthic
light substitution are detailed above in PM1.

Stan code (surface light):
\url{https://github.com/jrblaszczak/RiverBiomass/blob/main/code/Stan_ProductivityModel2_Logistic.stan}

Stan code (benthic light):
\url{https://github.com/jrblaszczak/RiverBiomass/blob/main/code/Stan_ProductivityModel2_Logistic_BenthicL.stan}

Simulation code (surface and benthic light):
\url{https://github.com/jrblaszczak/RiverBiomass/blob/main/code/Simulated_ProductivityModel2_Logistic.R}

\hypertarget{productivity-model-3-latent-biomass-ricker-model}{%
\subsubsection{Productivity Model 3: Latent biomass Ricker
model}\label{productivity-model-3-latent-biomass-ricker-model}}

We predict the temporal dynamics of mean daily estimates of GPP (\(g\))
by incorporating biomass as a latent variable and modeling its dynamics
using a Ricker growth model. We fit the model to the \(g\) times series
from Appling et al.~(2018) to estimate the model parameters. We
incorporated the effects of disturbance by modeling the persistence
(\(P\)) of log biomass using a complementary log-log link function of
the same form as above.

Here we use the Ricker model to model the discrete time step dynamics of
log biomass (\(B_t\)), but we start with the original form with biomass
(\(b_t\)):

\begin{equation}
    b_{t} = b_{(t-\Delta t)}e^{r_{max}(1-\frac{b_{(t-\Delta t)}}{K})}
\end{equation}

Then if we take the log of both sides of the equation and multiply
\(r_{max}\) through,

\begin{equation}
    log(b_{t}) = log(b_{(t-\Delta t)})+r_{max}-\frac{r_{max}b_{(t-\Delta t)}}{K}
\end{equation}

Next, if we substitute \(B_t = log(b_t)\) and
\(\lambda = \frac{-r_{max}}{K}\), then we arrive at the final form used
here:

\begin{equation}
    B_{t} \sim N(P_{t}(B_{(t-\Delta t)} + r_{max} + \lambda e^{B_{(t-\Delta t)}}), \sigma_{proc})
\end{equation}

\begin{equation}
    g_{t} \sim N(L_{t}e^{B_{t}},\sigma_{obs})
\end{equation}

where estimated parameters include \(B_{t}\) which is a latent variable
representative of photosynthetically-active biomass on a natural log
scale and daily time step, \(P_{t}\) which is the daily persistence of
biomass dependent on hydrologic disturbance and removal detailed above,
\(K\) which is the estimated carrying capacity of a river, \(r_{max}\)
is the maximum per capita growth rate when \(\frac {B_t}{K}\) approaches
zero following a disturbance, \(\sigma_{obs}\) which is the observation
error set to the standard deviation of the posterior probability
distribution of the previously modeled daily GPP estimates (\(g\)),
\(\sigma_{proc}\) which is the process error. Input data and the benthic
light substitution are detailed above in PM1.

Stan code (surface light):
\url{https://github.com/jrblaszczak/RiverBiomass/blob/main/code/Stan_ProductivityModel3_Ricker.stan}

Stan code (benthic light):
\url{https://github.com/jrblaszczak/RiverBiomass/blob/main/code/Stan_ProductivityModel3_Ricker_BenthicL.stan}

Simulation code (surface and benthic light):
\url{https://github.com/jrblaszczak/RiverBiomass/blob/main/code/Simulated_ProductivityModel3_Ricker.R}

\hypertarget{productivity-model-4-latent-biomass-ricker-model-with-light-adjustment}{%
\subsubsection{Productivity Model 4: Latent biomass Ricker model with
light
adjustment}\label{productivity-model-4-latent-biomass-ricker-model-with-light-adjustment}}

We test a modification of the Ricker model (PM3) in which light is
incorporated into the estimation of maximum growth rate \(r_{max}\):

\begin{equation}
    B_{t} \sim N(P_{t}(B_{(t-\Delta t)} + r_{max,t} + \gamma e^{(B_{(t-\Delta t)})}), \sigma_{proc})
\end{equation}

\begin{equation}
    r_{max,t} = \alpha_{1}L_{t}
\end{equation}

\begin{equation}
    g_{t} \sim N(e^{B_{t}},\sigma_{obs})
\end{equation}

where estimated parameters, input data, and benthic light are the same
as in PM3 detailed above, except for \(\alpha_1\) which is a light use
efficiency parameter describing log biomass growth per unit light.

Stan code (surface light):
\url{https://github.com/jrblaszczak/RiverBiomass/blob/main/code/Stan_ProductivityModel4_Ricker_lightadj.stan}

Stan code (benthic light):
\url{https://github.com/jrblaszczak/RiverBiomass/blob/main/code/Stan_ProductivityModel4_Ricker_lightadj_BenthicL.stan}

Simulation code (surface and benthic light):
\url{https://github.com/jrblaszczak/RiverBiomass/blob/main/code/Simulated_ProductivityModel4_Ricker_lightadj.R}

\hypertarget{productivity-model-5-gompertz}{%
\subsubsection{Productivity Model 5:
Gompertz}\label{productivity-model-5-gompertz}}

We predict the temporal dynamics of mean daily estimates of GPP (\(g\))
by incorporating biomass as a latent variable and modeling its dynamics
using a Gompertz growth model. We fit the model to the \(g\) times
series from Appling et al.~(2018) to estimate the model parameters. We
incorporated the effects of disturbance by modeling the persistence
(\(P\)) of biomass using a complementary log-log link function of the
same form as above.

Here we use the Gompertz model where biomass is on a log scale detailed
in:

Ives, A.R., Dennis, B., Cottingham, K.L. \& Carpenter, S.R. (2003).
Estimating community stability and ecological interactions from
time-series data. Ecol. Monogr., 73, 301-330.

We add light as a covariate to the model to describe growth:

\begin{equation}
    B_{t} \sim N(P_{t}(\beta_{0} + \beta_{1}e^{(B_{(t-\Delta t)})} + \beta_{2}L_{t}), \sigma_{proc})
\end{equation}

\begin{equation}
    g_{t} \sim N(e^{B_{t}},\sigma_{obs})
\end{equation}

where estimated parameters include \(B_{t}\) which is a latent variable
representative of photosynthetically-active biomass on a natural log
scale and daily time step, \(P_{t}\) which is the daily persistence of
biomass dependent on hydrologic disturbance and removal detailed above,
\(\beta_0\) which is the intrinsic rate of increase, \(\beta_1\) governs
the strength of density dependence, \(\beta_2\) which is a light use
efficiency parameter, \(\sigma_{obs}\) which is the observation error
set to the standard deviation of the posterior probability distribution
of the previously modeled daily GPP estimates (\(g\)), and
\(\sigma_{proc}\) which is the process error. Input data and the benthic
light substitution are detailed above in PM1.

Stan code (surface light):
\url{https://github.com/jrblaszczak/RiverBiomass/blob/main/code/Stan_ProductivityModel5_Gompertz.stan}

Stan code (benthic light):
\url{https://github.com/jrblaszczak/RiverBiomass/blob/main/code/Stan_ProductivityModel5_Gompertz_BenthicL.stan}

Simulation code (surface and benthic light):
\url{https://github.com/jrblaszczak/RiverBiomass/blob/main/code/Simulated_ProductivityModel5_Gompertz.R}

\hypertarget{model-simulation-comparisons}{%
\subsection{2. Model simulation
comparisons}\label{model-simulation-comparisons}}

df\_sim1\_plot

\hypertarget{model-comparison-across-9-rivers}{%
\subsection{3. Model comparison across 9
rivers}\label{model-comparison-across-9-rivers}}

\end{document}
